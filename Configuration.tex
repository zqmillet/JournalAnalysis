%!TEX root = ./Main.tex

\usepackage{amsmath}
\usepackage{mathrsfs}
\usepackage{mathtools}
\usepackage{commath}
\usepackage{amssymb}

\usepackage{bm}
\usepackage{tabu, longtable}
\usepackage{color}
\usepackage{tikz, pgfplots, tkz-euclide}
\usetikzlibrary{calc}
\usetikzlibrary{shadows}
\usetikzlibrary{positioning}
\usetikzlibrary{circuits.ee.IEC}
\usetikzlibrary{decorations.pathmorphing}
\usetikzlibrary{shapes.symbols}
\usetikzlibrary{shapes.geometric}
\usetikzlibrary{intersections}
\usetikzlibrary{spy}
\usepgfplotslibrary{fillbetween}
\usetkzobj{all}
\pgfplotsset{compat=1.12}


\usepackage{ifthen}
\usepackage{subfigure}
\usepackage{float}
\usepackage{cleveref}
\crefname{equation}{equation}{equations}
\Crefname{equation}{Equation}{Equations}
\crefname{section}{Section}{Sections}
\Crefname{section}{Section}{Sections}
\crefname{figure}{Fig.}{Figures}
\Crefname{figure}{Fig.}{Figures}
\crefname{table}{Table}{Tables}
\Crefname{table}{Table}{Tables}
\crefname{algorithm}{Algorithm}{Algorithms}
\Crefname{algorithm}{Algorithm}{Algorithms}
\crefname{prop}{Proposition}{Propositions}
\Crefname{prop}{Proposition}{Propositions}
\crefname{algorithm}{Algorithm}{Algorithms}
\Crefname{algorithm}{Algorithm}{Algorithms}

\usepackage{keycommand}
\usepackage{enumerate}
\usepackage{geometry}
\usepackage{algorithm}
\usepackage{algorithmic}
\renewcommand{\algorithmicrequire}{\textbf{Input:}}
\renewcommand{\algorithmicensure}{\textbf{Output:}}
\renewcommand{\algorithmicforall}{\textbf{for each}}
\newcommand{\algorithmiccontinue}{\textbf{continue for}}
\newcommand{\CONTINUE}{\STATE \algorithmiccontinue}

% If don't want to use the custom style, use the command \UseCustomStylefalse.
\newif\ifUseCustomStyle
\UseCustomStyletrue

\ifUseCustomStyle
    \linespread{1.66}

    \newgeometry{left=1in,right=1in,top=1in,bottom=1in}
    \usepackage{caption}
    \captionsetup[figure]{
        font = footnotesize,
        labelsep = period,
        textformat = period,
        skip = 0pt
    }

    \captionsetup[table]{
        font = {footnotesize, sc},
        labelsep = newline,
        skip = 0pt
    }

    \extrarowsep = -2pt
\fi

% The following codes are used to draw help lines for tikzpicture.
\makeatletter
\def\grd@save@target#1{%
  \def\grd@target{#1}}
\def\grd@save@start#1{%
  \def\grd@start{#1}}
\tikzset{
  grid with coordinates/.style={
    to path={%
      \pgfextra{%
        \edef\grd@@target{(\tikztotarget)}%
        \tikz@scan@one@point\grd@save@target\grd@@target\relax
        \edef\grd@@start{(\tikztostart)}%
        \tikz@scan@one@point\grd@save@start\grd@@start\relax
        \draw[minor help lines] (\tikztostart) grid (\tikztotarget);
        \draw[major help lines] (\tikztostart) grid (\tikztotarget);
        \grd@start
        \pgfmathsetmacro{\grd@xa}{\the\pgf@x/1cm}
        \pgfmathsetmacro{\grd@ya}{\the\pgf@y/1cm}
        \grd@target
        \pgfmathsetmacro{\grd@xb}{\the\pgf@x/1cm}
        \pgfmathsetmacro{\grd@yb}{\the\pgf@y/1cm}
        \pgfmathsetmacro{\grd@xc}{\grd@xa + \pgfkeysvalueof{/tikz/grid with coordinates/major step}}
        \pgfmathsetmacro{\grd@yc}{\grd@ya + \pgfkeysvalueof{/tikz/grid with coordinates/major step}}
        \foreach \x in {\grd@xa,\grd@xc,...,\grd@xb}
        \node[anchor=north] at (\x,\grd@ya) {\pgfmathprintnumber{\x}};
        \foreach \y in {\grd@ya,\grd@yc,...,\grd@yb}
        \node[anchor=east] at (\grd@xa,\y) {\pgfmathprintnumber{\y}};
      }
    }
  },
  minor help lines/.style={
    help lines,
    step=\pgfkeysvalueof{/tikz/grid with coordinates/minor step}
  },
  major help lines/.style={
    help lines,
    line width=\pgfkeysvalueof{/tikz/grid with coordinates/major line width},
    step=\pgfkeysvalueof{/tikz/grid with coordinates/major step}
  },
  grid with coordinates/.cd,
  minor step/.initial=.2,
  major step/.initial=1,
  major line width/.initial=2pt,
}
\makeatother

\tikzset{circuit declare symbol = ac source}
\tikzset{font = \fontsize{10pt}{12pt}\selectfont}
\tikzset{set ac source graphic = ac source IEC graphic}
\tikzset
{
    ac source IEC graphic/.style=
    {
        transform shape,
        circuit symbol lines,
        circuit symbol size = width 3 height 3,
        shape=generic circle IEC,
        /pgf/generic circle IEC/before background=
        {
            \pgfpathmoveto{\pgfpoint{-0.8pt}{0pt}}
            \pgfpathsine{\pgfpoint{0.4pt}{0.4pt}}
            \pgfpathcosine{\pgfpoint{0.4pt}{-0.4pt}}
            \pgfpathsine{\pgfpoint{0.4pt}{-0.4pt}}
            \pgfpathcosine{\pgfpoint{0.4pt}{0.4pt}}
            \pgfusepathqstroke
        }
    }
}

\tikzset{
	text shadow/.code args={[#1]#2at#3(#4)#5}{
		\pgfkeysalso{/tikz/.cd,#1}%
		\foreach \angle in {0,5,...,359}{
			\node[#1,text=white] at ([shift={(\angle:.8pt)}] #4){#5};
		}
	}
}

\newtheorem{prop}{Proposition}

% The follow codes are macros for frequent terms.
\newcommand{\etal}{\emph{et al}.}
\newcommand{\risk}{\mathscr{R}}
\newcommand{\pu}[1]{\overline{#1}{}} % probability upper
\newcommand{\pl}[1]{\underline{#1}{}} % probability lower
\newcommand{\fp}[1]{\widetilde{#1}{}} % fuzzy probability
\newkeycommand{\fpe}[symbol = p][1]{ % fuzzy probability expansion
    (\pl{\commandkey{symbol}}_{#1}, \commandkey{symbol}_{#1}, \pu{\commandkey{symbol}}_{#1})
}
\newcommand{\acuts}{$\alpha$-cuts{}}
\newcommand{\acut}[1]{{}^{\alpha}\hspace{-1.8pt}{#1}}
\newcommand{\abfbai}{\acuts{} based fuzzy Bayesian approximate inference}
\newcommand{\Abfbai}{\acuts{} Based Fuzzy Bayesian Approximate Inference}
\newcommand{\lthreshold}{D_{\min}}
\newcommand{\bnt}{\mathscr{B}}
\newcommand{\cmodel}{\mathscr{C}}
\newcommand{\pnode}[1]{{}^{*}\hspace{-1.4pt}{#1}}
\newcommand{\cnode}[1]{{#1}^*}
\newcommand{\var}{\hspace{1pt}\cdot\hspace{1pt}} 