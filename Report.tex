%!TEX program = xelatex
%!TEX builder = latexmk

\documentclass[12pt,
               a4paper,
               onecolumn]{article}

%!TEX root = ./Main.tex

\usepackage{amsmath}
\usepackage{mathrsfs}
\usepackage{mathtools}
\usepackage{commath}
\usepackage{amssymb}

\usepackage{bm}
\usepackage{tabu, longtable}
\usepackage{multirow}
\usepackage{float}
\usepackage[hidelinks]{hyperref}
\usepackage{color}
\usepackage{tikz, pgfplots}
\pgfplotsset{compat=1.12}


\usepackage{cleveref}
\crefname{equation}{equation}{equations}
\Crefname{equation}{Equation}{Equations}
\crefname{section}{Section}{Sections}
\Crefname{section}{Section}{Sections}
\crefname{figure}{Fig.}{Figures}
\Crefname{figure}{Fig.}{Figures}
\crefname{table}{Table}{Tables}
\Crefname{table}{Table}{Tables}
\crefname{algorithm}{Algorithm}{Algorithms}
\Crefname{algorithm}{Algorithm}{Algorithms}
\crefname{prop}{Proposition}{Propositions}
\Crefname{prop}{Proposition}{Propositions}
\crefname{algorithm}{Algorithm}{Algorithms}
\Crefname{algorithm}{Algorithm}{Algorithms}

\usepackage{enumitem}
\setlist[1]{itemsep=-5pt}
\usepackage{geometry}


% If don't want to use the custom style, use the command \UseCustomStylefalse.
\newif\ifUseCustomStyle
\UseCustomStyletrue

\ifUseCustomStyle
    \linespread{1.66}

    \newgeometry{left=1in,right=1in,top=1in,bottom=1in}
    \usepackage{caption}
    \captionsetup[figure]{
        font = footnotesize,
        labelsep = period,
        textformat = period,
        skip = 0pt
    }

    \captionsetup[table]{
        font = {footnotesize, sc},
        labelsep = period,
        skip = 0pt
    }

    \extrarowsep = -2pt
\fi

\newcommand{\myquote}[1]{
  \begingroup
    \tabulinesep = 10pt
    \begin{tabu}to \textwidth{|[5pt, blue]X}
      #1
    \end{tabu}
  \endgroup
}

\newcommand{\ReviewTimeDistribution}[2]{
  The review time distribution of #2 is shown in \cref{fig:Review Time Distribution of #1}.
  \begin{figure}[htb]
    \centering
    \begin{tikzpicture}
      \begin{axis}[bar width   = 0.36cm,
                   width       = 8cm,
                   height      = 5.8cm,
                   ylabel      = Paper Number,
                   xlabel      = Review Time (Day),
                   label style = {font = \footnotesize},
                  ]
        \addplot[ybar, fill = black, draw = none] file {./Data/ReviewTimeDistribution/#1.dat};
      \end{axis}
    \end{tikzpicture}
    \caption{Review Time Distribution of #2}
    \label{fig:Review Time Distribution of #1}
  \end{figure}
}

\newcommand{\PageNumberDistribution}[2]{
  The page number distribution of #2 is shown in \cref{fig:Review Time Distribution of #1}.
  \begin{figure}[htb]
    \centering
    \begin{tikzpicture}
      \begin{axis}[bar width   = 0.55cm,
                   width       = 8cm,
                   height      = 5.8cm,
                   ylabel      = Paper Number,
                   xlabel      = Page Number,
                   label style = {font = \footnotesize},
                  ]
        \addplot[ybar, fill = black, draw = none] file {./Data/PageNumberDistribution/#1.dat};
      \end{axis}
    \end{tikzpicture}
    \caption{Page Number Distribution of #2}
    \label{fig:Page Number Distribution of #1}
  \end{figure}
}

\newcommand{\RelationshipBetweenRTPN}[2]{
  The relationship between review time and page number of #2 is shown in \cref{fig:Relationship between Review Time and Page Number of #1}.
  \begin{figure}[htb]
    \centering
    \begin{tikzpicture}
      \begin{axis}[width       = 8cm,
                   height      = 5.8cm,
                   ylabel      = Page Number,
                   xlabel      = Review Time (Day),
                   label style = {font = \footnotesize}]
        \addplot[scatter, only marks, scatter/use mapped color = 
        {draw opacity = 0, fill = mapped color}] table[x index = 0, y index = 1, col sep=space] {./Data/RelationshipBetweenReviewTimeAndPageNumber/#1.dat};
      \end{axis}
    \end{tikzpicture}
    \caption{Relationship between Review Time and Page Number of #2}
    \label{fig:Relationship between Review Time and Page Number of #1}
  \end{figure}
}

\newcommand{\POTI}{\href{http://ieeexplore.ieee.org/xpl/RecentIssue.jsp?punumber=5}{Proceedings of the IEEE}}
\newcommand{\TII}{\href{http://ieeexplore.ieee.org/xpl/RecentIssue.jsp?punumber=9424}{IEEE Transactions on Industrial Informatics}}
\newcommand{\TIE}{\href{http://ieeexplore.ieee.org/xpl/RecentIssue.jsp?punumber=41}{IEEE Transactions on Industrial Electronics}}
\newcommand{\TIFS}{\href{http://ieeexplore.ieee.org/xpl/RecentIssue.jsp?punumber=10206}{IEEE Transactions on Information Forensics and Security}}
\renewcommand{\SS}{\href{http://www.journals.elsevier.com/safety-science}{Safety Science}}
\newcommand{\ARIC}{\href{http://www.journals.elsevier.com/annual-reviews-in-control}{Annual Reviews in Control}}

\newcommand{\ReviewTimeDetail}[3]{
  \begin{itemize}
    \item The minimum review time is #1 (day),
    \item The average review time is #2 (day),
    \item The maximum review time is #3 (day).
  \end{itemize}
}

\newcommand{\PageNumberDetail}[3]{
  \begin{itemize}
    \item The minimum page number is #1,
    \item The average page number is #2,
    \item The maximum page number is #3.
  \end{itemize}
}

\begin{document}
\section{Analysis of \POTI{}}
\myquote{
  The most highly-cited general interest journal in electrical engineering and computer science, the \POTI{} is the best way to stay informed on an exemplary range of topics. This journal also holds the distinction of having the longest useful archival life of any EE or computer related journal in the world! Since 1913, the \POTI{} has been the leading journal to provide in-depth tutorial and review coverage of the technical developments that shape our world.
}

\begin{table}[htb]
  \centering
  \caption{Detail of \POTI{}}
  \label{fig:Detail of POTE}
  \begin{tabu}to 0.8\textwidth{*4{X[c]}}
  \tabucline[1pt]{-}
    Year & Impact Factor & Total Articles & Total Cites\\
  \hline
    2014/2015 & 4.934    & 102            & 21017 \\
    2013      & 5.466    & 154            & 20916 \\
    2012      & 6.911    & 195            & 18840 \\
    2011      & 6.810    & 118            & 16872 \\
    2010      & 5.096    & 139            & 16971 \\
    2009      & 4.878    & 129            & 17919 \\
    2008      & 4.613    & 122            & 17993 \\
  \tabucline[1pt]{-}
  \end{tabu}
\end{table}

%!TEX root = ../../Main.tex

There are 183 papers of \POTI{} for statistics.

\ReviewTimeDistribution{ProceedingsOfTheIEEE}{\POTI{}}
\ReviewTimeDetail{15}{162.7923}{564}
\PageNumberDistribution{ProceedingsOfTheIEEE}{\POTI{}}
\PageNumberDetail{7}{17.4809}{63}
\RelationshipBetweenRTPN{ProceedingsOfTheIEEE}{\POTI{}}

\section{Analysis of \TIE{}}
\myquote{
  \TIE{} is published monthly. Its scope encompasses the applications of electronics, controls and communications, instrumentation and computational intelligence for the enhancement of industrial and manufacturing systems and processes. Included are power electronics and drive control techniques, system control and signal processing, fault detection and diagnosis, power systems, instrumentation, measurement and testing, modeling and simulation, motion control, robotics, sensors and actuators, implementation of neural nets, fuzzy logic, and artificial intelligence in industrial systems, factory automation, communication, and computer networks.
}

\begin{table}[htb]
  \centering
  \caption{Detail of \TIE{}}
  \label{fig:Detail of TIE}
  \begin{tabu}to 0.8\textwidth{*4{X[c]}}
  \tabucline[1pt]{-}
    Year & Impact Factor & Total Articles & Total Cites\\
  \hline
    2014/2015 & 6.498    & 694            & 27141  \\
    2013      & 6.500    & 553            & 24432  \\
    2012      & 5.165    & 470            & 17404  \\
    2011      & 5.160    & 531            & 15474  \\
    2010      & 3.439    & 434            & 10294  \\
    2009      & 4.678    & 505            & 10306  \\
    2008      & 5.468    & 454            & 9014   \\
  \tabucline[1pt]{-}
  \end{tabu}
\end{table}

%!TEX root = ../../Main.tex

There are 310 papers of \TIE{} for statistics.

\ReviewTimeDistribution{IEEETransactionOnIndustrialElectronics}{\TIE{}}
\ReviewTimeDetail{11}{204.7581}{597}
\PageNumberDistribution{IEEETransactionOnIndustrialElectronics}{\TIE{}}
\PageNumberDetail{3}{10.5774}{16}
\RelationshipBetweenRTPN{IEEETransactionOnIndustrialElectronics}{\TIE{}}

\section{Analysis of \TII{}}
\myquote{
  \TII{} focuses on knowledge-based factory automation as a means to enhance industrial fabrication and manufacturing processes. This embraces a collection of techniques that use information analysis, manipulation, and distribution to achieve higher efficiency, effectiveness, reliability, and/or security within the industrial environment. The scope of the Transaction includes reporting, defining, providing a forum for discourse, and informing its readers about the latest developments in intelligent and computer control systems, robotics, factory communications and automation, flexible manufacturing, visionsystems, and data acquisition and signal processing.
}

\begin{table}[htb]
  \centering
  \caption{Detail of \TII{}}
  \label{fig:Detail of TII}
  \begin{tabu}to 0.8\textwidth{*4{X[c]}}
  \tabucline[1pt]{-}
    Year & Impact Factor & Total Articles & Total Cites\\
  \hline
    2014/2015 & --       & --             & --    \\
    2013      & 8.785    & 231            & 2644 \\
    2012      & 3.381    & 92             & 969  \\
    2011      & 2.990    & 71             & 739  \\
    2010      & 1.627    & 63             & 328  \\
    2009      & 1.614    & 39             & 287  \\
    2008      & 2.356    & 28             & 227  \\
  \tabucline[1pt]{-}
  \end{tabu}
\end{table}

%!TEX root = ../../Main.tex

There are 413 papers of \TII{} for statistics.

\ReviewTimeDistribution{IEEETransactionOnIndustrialInformatics}{\TII{}}
\ReviewTimeDetail{1}{245.3293}{877}
\PageNumberDistribution{IEEETransactionOnIndustrialInformatics}{\TII{}}
\PageNumberDetail{5}{11.4092}{18}
\RelationshipBetweenRTPN{IEEETransactionOnIndustrialInformatics}{\TII{}}

\section{Analysis of \TIFS{}}
\myquote{
  \TIFS{} covers the sciences, technologies, and applications relating to information forensics, information security, biometrics, surveillance and systems applications that incorporate these features.
}

\begin{table}[htb]
  \centering
  \caption{Detail of \TIFS{}}
  \label{fig:Detail of TIFS}
  \begin{tabu}to 0.8\textwidth{*4{X[c]}}
  \tabucline[1pt]{-}
    Year & Impact Factor & Total Articles & Total Cites\\
  \hline
    2014/2015 & 2.408    & 144            & 2376 \\
    2013      & 2.065    & 176            & 1598 \\
    2012      & 1.895    & 157            & 1134 \\
    2011      & 1.340    & 119            & 759  \\
    2010      & 1.725    & 81             & 682  \\
    2009      & 2.338    & 81             & 540  \\
    2008      & 2.230    & 68             & 265  \\
  \tabucline[1pt]{-}
  \end{tabu}
\end{table}

%!TEX root = ../../Main.tex

There are 123 papers of \TIFS{} for statistics.

\ReviewTimeDistribution{IEEETransactionOnInformationForensicsAndSecurity}{\TIFS{}}
\ReviewTimeDetail{16}{196.0488}{428}
\PageNumberDistribution{IEEETransactionOnInformationForensicsAndSecurity}{\TIFS{}}
\PageNumberDetail{1}{12.9512}{27}
\RelationshipBetweenRTPN{IEEETransactionOnInformationForensicsAndSecurity}{\TIFS{}}

\section{Analysis of \SS{}}
\myquote{
  \SS{} serves as an international medium for research in the science and technology of human safety. It extends from safety of people at work to other spheres, such as transport, leisure and home, as well as every other field of man is hazardous activities.\\
  \SS{} is multidisciplinary. Its contributors and its audience range from psychologists to chemical engineers. The journal covers the physics and engineering of safety; its social, policy and organisational aspects; the management of risks; the effectiveness of control techniques for safety; standardization, legislation, inspection, insurance, costing aspects, human behaviour and safety and the like.
}

\begin{table}[htb]
  \centering
  \caption{Detail of \SS{}}
  \label{fig:Detail of SS}
  \begin{tabu}to 0.8\textwidth{*4{X[c]}}
  \tabucline[1pt]{-}
    Year & Impact Factor & Total Articles & Total Cites\\
  \hline
    2014/2015 & 1.831    & 230            & 3959 \\
    2013      & 1.672    & 221            & 3181 \\
    2012      & 1.359    & 246            & 2393 \\
    2011      & 1.402    & 159            & 1786 \\
    2010      & 1.637    & 175            & 1788 \\
    2009      & 1.220    & 153            & 1274 \\
    2008      & 0.836    & 114            & 921  \\
  \tabucline[1pt]{-}
  \end{tabu}
\end{table}

%!TEX root = ../../Main.tex

There are 308 papers of \SS{} for statistics.

\ReviewTimeDistribution{SafetyScience}{\SS{}}
\ReviewTimeDetail{19}{248.1364}{972}
\PageNumberDistribution{SafetyScience}{\SS{}}
\PageNumberDetail{4}{10.3766}{25}
\RelationshipBetweenRTPN{SafetyScience}{\SS{}}

\section{Analysis of \ARIC{}}
\myquote{
  \ARIC{} covers the whole field of control and its applications. Most reviews are selected from the best reviews presented at meetings of IFAC, the International Federation of Automatic Control, re-written and broadened where necessary. The journal also seeks to commission reviews in emerging research areas from leading experts. Suggestions for new review articles should be sent to the Editor or to a member of the Editorial Board. Principal topics include nonlinear control, stochastic theory, discrete events, linear systems, adaptive control, robust control, design and software, system identification, fault detection, real-time programming, robot control, artificial intelligence, man-machine systems, optimization, computer-aided design and intelligent control.
}

\begin{table}[htb]
  \centering
  \caption{Detail of \ARIC{}}
  \label{fig:Detail of ARIC}
  \begin{tabu}to 0.8\textwidth{*4{X[c]}}
  \tabucline[1pt]{-}
    Year & Impact Factor & Total Articles & Total Cites\\
  \hline
    2014/2015 & 2.518    & 20             & 957 \\
    2013      & 1.878    & 28             & 788 \\
    2012      & 1.289    & 28             & 662 \\
    2011      & 1.319    & 21             & 482 \\
    2010      & 1.884    & 24             & 410 \\
    2009      & 1.886    & 23             & 441 \\
    2008      & 1.109    & 20             & 365 \\
  \tabucline[1pt]{-}
  \end{tabu}
\end{table}

%!TEX root = ../../Main.tex

There are 102 papers of \ARIC{} for statistics.

\ReviewTimeDistribution{AnnualReviewsInControl}{\ARIC{}}
\ReviewTimeDetail{2}{120.4706}{727}
\PageNumberDistribution{AnnualReviewsInControl}{\ARIC{}}
\PageNumberDetail{7}{13.2255}{29}
\RelationshipBetweenRTPN{AnnualReviewsInControl}{\ARIC{}}

\section{Summary}

\begin{figure}[H]
  \small
  \begin{tabu}to \textwidth{X[1]*3{X[c, -1]}}
    \tabucline[1pt]{-}
    \multirow{2}{*}{Journal} & \multicolumn{3}{c}{Review Time (day)}\\\cline{2-4}
     & Minimin & Average & Minimum\\ 
    \hline
    %!TEX root = ../../Main.tex

\POTI{} & 15 & 162.7923 & 564 \\
\TIE{} & 11 & 204.7581 & 597 \\
\TII{} & 1 & 245.3293 & 877 \\
\TIFS{} & 16 & 196.0488 & 428 \\
\SS{} & 19 & 248.1364 & 972 \\
\ARIC{} & 2 & 120.4706 & 727 \\
    \tabucline[1pt]{-}
  \end{tabu}
\end{figure}

\begin{figure}[H]
  \small
  \begin{tabu}to \textwidth{X[1]*3{X[c, -1]}}
    \tabucline[1pt]{-}
    \multirow{2}{*}{Journal} & \multicolumn{3}{c}{Page Number}\\\cline{2-4}
     & Minimin & Average & Minimum\\ 
    \hline
    %!TEX root = ../../Main.tex

\POTI{} & 7 & 17.4809 & 63 \\
\TIE{} & 3 & 10.5774 & 16 \\
\TII{} & 5 & 11.4092 & 18 \\
\TIFS{} & 1 & 12.9512 & 27 \\
\SS{} & 4 & 10.3766 & 25 \\
\ARIC{} & 7 & 13.2255 & 29 \\
    \tabucline[1pt]{-}
  \end{tabu}
\end{figure}

\end{document}

